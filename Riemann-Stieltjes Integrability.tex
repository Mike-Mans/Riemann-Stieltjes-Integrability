\documentclass{report} 

\usepackage{amsmath} 
\usepackage{graphicx}
\usepackage{xcolor}

\title{Riemann–Stieltjes Integrability}
\author{Michael Mansour}
\date{\today}
\begin{document}
\maketitle

\section{Motivation}
Previously, we have discussed and defined Riemann Sums and Integrability, 
and this can intuitively be thought of as a sum of a continuous function, 
f, with respect to a partition along x, where $P = \{t_i  \in [a,b]\}$. We would like to extend 
this concept of summation but now with respect to another function, F(x).
This has many different applications, and we will discuss the applications
to probability.
\subsection{Random Variable, PDF, CDF, Motivating Example}
A random variable is a function that assigns events in the space of all outcomes
to probabilities associated with those outcomes. For example, a coin flip could map
X=1(heads) to 1/2 and X=0(tails) to 1/2. A random variable has an associated PDF (the probability 
that $X = x$ , $P(X=x)$) and CDF (the probability that $X <= x$, $P(X <= x)$). It's also a fact that
the sum of all possible pdf values across all different events must equal one (e.g. P(X=0) + 
P(X=1) = 1/2 + 1/2 = 1)

So, consider a random variable with a pdf (namely F'(x)), defined as follows:
\[
    F'(x) = 
    \begin{cases}
        1/5 \text{ if } x \in [0,1) \\
        4/5 \text{ if } x = 1
    \end{cases}
\]

\begin{center}
    \includegraphics[width=1\textwidth]{PDF_CDF.png}
\end{center}

Now, to consider taking a sum of all possible values, we must consider integration
since we are discussing a continous interval from [0,1). So, we must integrate the function
and consider the appropriate weight.
\newline \newline
If we are considering integration, we should first ask if this is Riemann-Integrable
as discussed in class. It is Riemann-Integrable because the function is only discontinuous
at a finite number of points (namely when x = 1). So, we can integrate the pdf and arrive 
at $\int_{0}^{1} F'(x) = 1/5$ since the integral excludes finite discontinuities from the final result. 
However, this contradicts the stated fact that the sum of all pdf values 
must equal one. So, we must consider an improved method of integration that
can account for the weight applied at x=1 (namely adding in that extra weight of 4/5 to achieve
a final result where $\int_{0}^{1} F'(x) = 1$)

\section{Definitions and Statements}

\subsection{Darboux–Stieltjes Integrable Definition:}
Let F be an increasing function on a closed interval [a,b] and \( F(a)
< F(b) \). F may have jumps and be discontinuous. Let f be a bounded 
function on [a,b].
\newline \newline
Recall that the Upper Darboux Sum for a partition $P = \{t_0 < ... < t_i < ... < t_n\}$ is  
$U(f,P) = \sum_{i=1}^{n} sup\{f(x): x \in [t_{i-1}, t_i] \}(t_i - t_{i-1}).$ If we wanted to generalize
this to a summation with respect to F(x), we can rewrite this as $\sum_{i=1}^{n} sup\{f(x): x \in [t_{i-1}, t_i] \}(F(t_i) - F(t_{i-1}))$. However, 
this does not account for discontinuities/jumps we previously allowed. So, we revise this and establish the following 
defintion that handles discontinuous jumps in F (Note that $F(x^-)$ is the left-sided
limit of F at x and $F(x^+)$ is the right-sided limit. This is necessary because if discontinuities exist,
then the limits may be undefined because left-sided and right-sided limits might not agree). Also, we will use 
$F(a^-) = F(a)$ and $F(b^+)=F(b)$ as the limits of the endpoints.
\[
    J_F(f, P) = \sum_{i=0}^{n} f(t_i)(F(t_i^+) - F(t_i^-))
\]
\[
    U_F(f, P) = J_F(f, P) + \sum_{i=1}^{n} sup\{f(x): x \in (t_{i-1}, t_i) \}(F(t_i^-) - F(t_{i-1}^+))
\]
\[
    L_F(f, P) = J_F(f, P) + \sum_{i=1}^{n} inf\{f(x): x \in (t_{i-1}, t_i) \}(F(t_i^-) - F(t_{i-1}^+))
\]
\newline \newline
These are called the Upper and Lower Darboux–Stieltjes Sums, we can define the Upper and Lower Darboux–Stieltjes 
Integrals as the infimum and supremem (respectively) of these sums of all possible partitions P. Similar to 
Darboux Integrability, we say f is Darboux–Stieltjes Integrable if $L_F(f) = U_F(f) = \int_{a}^{b}fdF$.
\newline \newline
Note 1: $J_F(f, P)$ represents the total weight of the jumps, if F has jumps at the chosen partition points $t_i$.
Then, naturally, if F is continuous, $J_F(f, P) = 0$ since $(F(t_i^+) - F(t_i^-))$ would always be 0, regardless
of how the partition is chosen. It is also important to note that F can be discontinuous and a partition P 
can be chosen such that $J_F(f, P)$ is still 0 (i.e. none of the chosen points $t_i$ are the points where jumps occur).
\newline \newline
Note 2: We are stating the defintion of Darboux Integrability, which is 
synonymous with the Riemann Integrability definiton (where the 
function f(x) is being evaluated at some arbitrary x in the interval 
between \( t_{i-1} \) and \( t_i \)). It is also important to notice that since
we are explicitly adding in the jumps/discontinuities, we must use open intervals 
when discussing $sup\{f(x): x \in (t_{i-1}, t_i) \}$ and $inf\{f(x): x \in (t_{i-1}, t_i) \}$. This was not the case
with Riemann-Integrability from the previous section, where the discontinuity at a finite 
number of points did not affect the final result of the summation/integral. In other words, we are
weighing the jumps as part of our summation unlike the previously defined version of Riemann integration, 
so we must not include those points when taking the summation of the $sups$ and $infs$ at those subinterval endpoints. 
\newline \newline
Note 3: Clearly, we are taking partitions on [a,b] and then applying $F(t_i)$ to those
paritition points. Then it is natural to ask what if F is not one-to-one, which
is possible if we are only restricting F to being increasing and not strictly
increasing. In this case, we have distinct $t_i$ and $t_{i-1}$ where $F(t_{i-1}) = F(t_i)$.
Intuitively, this is how the Riemann–Stieltjes accounts for the weights of jumps at distinct points. 
Consider the following graph depicting F(x) vs x. Then notice the instance from 
c to d where F is not one-to-one. Then, we can see from the definition above that
$(F(t_i^-) - F(t_{i-1}^+)) = 0$ whenever $ t_{i-1}, t_{i} \in (c,d)$. Then we 
consider $J_F(f, P)$ and determine that the only instances where $(F(t_i^+) - F(t_i^-)) \neq 0$
are $(F(c^+) - F(c^-))$ and $(F(d^+) - F(d^-))$, and so we see that the weight associated
with these intervals become $f(c)(F(c^+) - F(c^-))+ f(d)(F(d^+) - F(d^-))$. 
\newline
\begin{center}
    \includegraphics[width=1\textwidth]{One-To-One Example.png}
\end{center}
\subsection{Darboux–Stieltjes Integrability Theorem:}
Following from the definition of $U_F(f, P)$ and $L_F(f, P$) and the
assumptions about f and F, we say that f is Darboux-Stieltjes Integrable
(or F-Integrable) iff $\forall \varepsilon>0$, $\exists P$
such that $U_F(f, P) - L_F(f, P) < \varepsilon$.
\newline \newline
Note 1: This is very similar to the Darboux Integrability Theorem discussed in the course
but this time using Upper and Lower Darboux–Stieltjes Sums and Integrals. Many other theorems 
that apply for Darboux Integrals can also be extended for Darboux–Stieltjes Integrals 
(such as continuity implying integrability, monotonic implying integrability, 
linearity of integration, etc.).

\subsection{Theorem (If F is differentiable)}
If F is differentiable on [a,b] and F' is Riemann-Integrable on [a,b], 
then f (a bounded function on [a,b]) is F-Integrable iff $fF'$ is Riemann-Integrable.
Then we have $\int_{a}^{b}f(x)dF = \int_{a}^{b} f(x)F'(x)dx$. 
\newline \newline
Note 1: If F is not differentiable (such as when we are dealing with a step function
that is not continuous) then this theorem does not apply, but the integral $\int_{a}^{b}f(x)dF$
is still well-defined as long as F is an increasing function as previously stated. 

\subsection{Riemann–Stieltjes Integrable as a Generalization of Riemann Integrable}
Riemann–Stieltjes Integrability is a generalization of Riemann-integrability, and we can
retrieve Riemann-Integrability when we set F(x) = x. Then, because x is differentiable everywhere,
we can say that its derivative F'(x) = 1, which (after applying the previous theorem) 
leaves us with $\int_{a}^{b}f(x)dx$.

\section{Exercises}
Given the following graphs of the CDFs of random variables, answer the following 
questions.
\begin{center}
    \includegraphics[width=0.5\textwidth]{Exercise 1.png}
\end{center}
a. What is P(X = 1) = P(X = 2)? \newline
\textcolor{blue}{Since the only points when the CDF increases are x=1 and x=2, and since each has a 
jump of 1/2, then P(X = 1) = P(X = 2) = 1/2.} \newline

b. Can we apply theorem 0.2.3 with f(x) = x? \newline
\textcolor{blue}{$F_1$ is a step function with discontinuous jumps, so it is 
not differentiable at those points where jumps occur. Since it is not 
differentiable, we cannot apply the theorem.} \newline

c. The expectation value of a random variable with CDF function F(x) is 
given by $\int_{a}^{b}xdF$. In this case, a = 0, b = 1, and $F_1(x)$ is the 
given graph. What is the expectation value for the random variable?
\textcolor{blue}{We notice that because the function values at F are the same except
at a finite number of points (namely x = 1 and x = 2), we only apply the weights to those values. 
In other words, the difference of $(F(t_i^-) - F(t_{i-1}^+))$ will be 0 everywhere else, so only 
consider the weights of x = 1 and x = 2. Then we see that $\int_{0}^{1}xdF$ = 1(1/2) + 2(1/2) = 3/2.} \newline


\begin{center}
    \includegraphics[width=0.5\textwidth]{Exercise 2.png}
\end{center}
a. It is a known that the PDF of a random variable
is the derivative of its CDF (if the derivative exists). Given $F_2(x) = cx^2$ is differentiable 
and $\int_{0}^{1}F_2'(x)dx = 1$, what must c equal? \newline
\textcolor{blue}{Since $F_2(x) = cx^2$, we can say that $F_2'(x) = 2cx$. Since $\int_{0}^{1}2cx = 1$, it follows that c = 1.} \newline

b. Can we apply theorem 0.2.3 with f(x) = x? \newline
\textcolor{blue}{$F_2 = x^2$ is a quadratic function, and since x is differentiable and continous,
it follows that $x^2$ must be differentiable and continuous as well. Its derivative $F_2'(x) = 2x$ 
is also continous and differentiable. Since f(x) = x is continous, then $fF_2' = 2x^2$ is continous and 
therefore Riemann-Integrable. So the theorem does apply.} \newline

c. What is the expectation value for the random variable? In other words, 
evaluate $\int_{0}^{1}xdF.$ \newline
\textcolor{blue}{We first apply theorem 0.2.3, and we see that $\int_{0}^{1}xdF = \int_{0}^{1}xF_2'(x)dx = \int_{0}^{1}2x^2dx$ = 2/3.} \newline
\end{document}